\magnification=\magstep1
\parindent=0pt
\parskip=6pt
\font\big=cmbx12

% stolen from /usr/share/texmf/tex/plain/base/cwebmac.tex
\newif\ifpdftex
\ifx\pdfoutput\undefined \pdftexfalse \else\ifnum\pdfoutput=0 \pdftexfalse
\else \pdftextrue \pdfoutput=1 \input pdfcolor \let\setcolor\pdfsetcolor \fi\fi

\ifpdftex
% target is pdf
% \input supp-pdf.tex  % definition of pdfimage
%\def\centerfig#1#2{\centerline{\pdfimage width #1 {#2.pdf}}}
\def\centerfig#1#2{\centerline{\pdfximage width #1 {#2.pdf}\pdfrefximage\pdflastximage}}
\def\barefig#1#2{\pdfximage width #1 {#2.pdf}\pdfrefximage\pdflastximage}
% Can't explain why the following two lines are needed, but the settings in
% /usr/share/texmf-texlive/tex/generic/config/pdftexconfig.tex
% are both wrong (A4 instead of letter) and ignored.
\pdfpagewidth=8.5 truein
\pdfpageheight=11.0 truein
\else
% target is xdvi/eps
\input epsf
\def\centerfig#1#2{\centerline{\epsfxsize=#1\epsffile{#2.eps}}}
\def\barefig#1#2{\epsfxsize=#1\epsffile{#2.eps}}
\fi


\centerline{\big The Hows and Whys of SEL feedback for LLRF}
\centerline{Larry Doolittle, LBNL, February 2014}

At its simplest, LLRF control should be thought of as a negative
feedback loop, just like an op-amp except with complex numbers.
For SRF cavities, consideration of setup and off-normal
conditions always lead designs to be based on self-excited loop
(SEL) topology.  Merging the two, without specifying excessive FPGA
resources or computational delays, is an art.  This paper will
explain and develop one particular implementation, targeted at
LCLS-2.  At the conclusion, synthesizable HDL code will be shown
to work in a system simulation.

The only topic here is the computation in the baseband complex plane.
Downconversion, upconversion, and any linearity corrections are
not considered.  The plant is simplified to include primarily the cavity,
with its pole at {\it circa} 20\thinspace Hz, negative real, plus an imaginary
component representing detuning.  The system delay, on the order of
800\thinspace ns, is also an essential part of the plant.
I will also assume that ADCs on the order of 100\thinspace MS/s are
used, which after processing will produce at most one complex number
measurement per two clock cycles.  Thus the data rate is at most
50\thinspace MS/s of complex numbers.  The single-sample SNR of these
converters is further assumed to be around 70\thinspace dB.

The net effect of negative feedback is to extend the bandwidth of
the cavity, from the open loop value of $\sim$20\thinspace Hz, to a closed
loop value up in the $\sim$60\thinspace kHz range.  That implies a
proportional gain term of $\sim$3000.  Without band-limiting, such
a high gain transfers far too much noise from the input channel
(ADC and related terms) to the power amplifier.   Indeed, the simple
combination of -70 dB input SNR and 70 dB of gain gives as much noise
as signal in the output drive.

Drive at frequencies much beyond the closed-loop cavity bandwidth does
nothing useful, and can be eliminated by a first-order low-pass filter.
The choice of cutoff frequency can be optimized later.
For now, list it as 300\thinspace kHz, a factor of 5 higher
than the 0\thinspace dB loop crossing.  While this filter does nothing
to the close-in noise power density delivered to the output, at least
now the total noise power is only 2\% of the nominal drive.
This filter introduces an additional 530\thinspace ns delay to the
feedback loop, but the total (now 1330\thinspace ns) is still
consistent with a 0\thinspace dB crossing at 60\thinspace kHz.

Proportional-only feedback has the well-known defect of leaving
residual static error.  An additional integral feedback term can
fix that problem.  System stability requires that the integral feedback
term not significantly perturb the system at the 0\thinspace dB
loop crossing frequency, but only contribute at much lower frequencies.

\vfill\eject
The controller/plant abstraction as developed so far is shown here:

\centerfig{0.8\hsize}{intro}

What is not shown explicitly in this diagram is that the imaginary part
of the cavity pole varies with time.  Its motion is driven by microphonics,
helium pressure fluctuation, Lorentz forces, and the tuner subsystem.

The noise suppression performance of this system is easily analyzed
in the frequency domain with Laplace techniques.
In terms of the plant and controller gains
$$A = e^{-sT}{p_C\over p_C-s}$$
$$K = (K_P + K_I/s){p_L\over p_L-s}$$
the noise gain is simply
$$1\over 1 + AK$$
\vfill\eject
Punching these equations into Octave/Matlab,
and using $K_I = 2\pi \cdot 20\thinspace {\rm kHz} \cdot K_P$,
gives these frequency response curves:

\centerfig{0.9\hsize}{over}

Now the complications begin.  Adding features, in particular SEL mode for
turn-on and handling large detuning excursions, involves lengthy calculations.
In particular, it is considered normal to implement a SEL by converting
from rectangular to polar with a CORDIC, computing PI feedback, and
converting back to rectangular with another \hbox{CORDIC}.  Retuning the
feedback loops to account for these computational delays would reduce
overall feedback performance.

\hfill\eject
There are two key observations.  The first is that, in the small signal,
that intricate calculation must get exactly the same result as the $K_P$
construction in the simple diagram.  The differences arise in the large
signal cases, as can be generated by the $K_I$ term.  The second observation
is that the delay sensitivity of that $K_I$ term is much weaker than that
of the $K_P$ term, since the former only becomes important at lower frequencies.
That conclusion can be quickly borne out by computing the noise gain for
various {\it additional} delays in the $K_I$ term, shown here:

\centerfig{0.8\hsize}{delays}

So we have huge freedom -- easily 100 cycles -- to do all the arithmetic we
want in the FPGA, as long as it is limited to the $K_I$ path.

\hfill\eject

The important features of an SEL data path were clearly laid out in 1978,
although using analog terminology.  A modern rendition of the block diagram
in the digital domain, loosely following JLab's experience, is shown here:

\centerfig{0.8\hsize}{sel_quad}

Where SPR stands for stateful phase resolver, and it is only used in the
phase channel.  More on that later.

This diagram shows both proportional and integral terms.  The stability limit
for the proportional term will be substantially smaller than the limit for
the direct, non-CORDIC-based path.  It's included here primarily so that the
amplitude loop can be closed during free-running SEL operation.  For highest
performance, in final operation this proportional path will be disabled, and
the direct path will take over.

Pure cavity-resonance-following SEL mode is easily set up with the
above hardware by setting amplitude and phase $K_I$ to zero, and
using the amplitude loop saturation values to set the drive level.
The $Y$ input to the output CORDIC is set to zero at this point, and the
phase offset needs to be given the right value for SEL operation.

Closing the amplitude loop is easily accomplished by putting in moderate
values of $K_P$ and $K_I$, and providing some separation between the lower
and upper limits for drive amplitude.

With the amplitude stabilized, it's just as easy to close the phase loop.
Here the stateful phase detector comes into play, forcing its output to
$\pm\pi/2$ when the phase is spinning counterclockwise or clockwise.
Only when the frequency error approaches zero does the output follow the input.
Using the Y output CORDIC on the output of the phase loop gives natural
amplitude compensation for detuning, as discussed by Delayen.  In combination
with the stateful phase detector and moderate values for the phase $K_P$ and
$K_I$, phase lock is naturally obtained by opening up the saturation limits
on the phase channel.  If $X_{\rm NOM}$ is the on-resonance drive amplitude,
the peak amplitude used for phase lock (including off-frequency operation)
becomes $\sqrt{X_{\rm NOM}^2+Y_{\rm LIM}^2}$.

Show an Octave-based simulation of the process here?
% spxb1_app/base.m

Once closed loop operation is established, the direct (low-latency) feedback
path should be configured to replace the proportional gain terms used in the
CORDIC path.  This can give higher gains, and therefore better control of
disturbances.  The direct path will also tend to have less DSP noise.

%More precisely, when in the $+\pi/2$ clipping state, it waits for its input
%phase to make a transition from $[\pi/2,\pi)$ to $[0,\pi/2)$.  And when in
%the $-\pi/2$ clipping state, it waits for its input phase to make a transition
%from $[-\pi,\pi/2)$ to $[-\pi/2,0)$.

Of course, the implementation of this logic in the FPGA doesn't look much
like the abstract data path above.  In particular, when the final direct
proportional loop is in use, its setpoint and gain need to be set consistently
with the CORDIC path.  I therefore choose to use the module containing the
CORDIC to generate those values, time-multiplexing the computational units
(a key concept from DSP Tutorial III, LLRF13).

The block diagram of the feedback path, not including the low-pass filter,
looks like:

\centerfig{1.0\hsize}{boxes}

Where the X-Y PI/clip engine evaluates both the amplitude and phase terms
of the feedback path shown in the previous figure.

Not counting the output low-pass filter, this construction takes less than
10\% of a small FPGA (XC6SLX45).  It represents 618 lines (non-blank,
non-comment) of synthesizable Verilog HDL, much of which was re-used from
previous projects.
% (cat `ls *.v | grep -v _tb.v`; cd ~/llrf/git/common_hdl; cat cordicg.v serialize.v doublediff.v vectormul.v) | grep -v "^//" | grep -c .

Elaborate on combining the low pass filter with an $8\pi/9$ notch filter
at about -800\thinspace kHz.

These simulations all use a time step of exactly 10\thinspace ns.  While
there are good reasons (non-IQ reference) for thinking the ADC sample rate
will not be exactly 100\thinspace MHz, there are also good reasons for
thinking this estimate is not wrong by more than 20\%.

\vfill\eject
While there are still a few bugs in the system,
I can show a system simulation of the Verilog
turning on a virtual cavity.  This cavity has an open-loop bandwidth of
10\thinspace kHz and is detuned by 2\thinspace kHz.  The phase loop
is turned on by gradually opening up the quadrature drive limits between
$t=110$\thinspace $\mu$s and $t=126$\thinspace $\mu$s.
% exp((-20e3+5e3*i)*2*pi*20e-9)
% log(0.99749+0.00063i)/20e-9/(2*pi)

\centerfig{0.82\hsize}{sima_m}

\centerfig{0.82\hsize}{sima_p}

\vfill\eject
While this work (when complete) should be viewed as a significant
down payment on the required gateware,
the following items are deliberately not considered here:
\def\i{\hfil\break\strut\quad$\circ$~}
\i Down- and up-conversion
\i Phase calibration
\i Input channel distortion correction (optional)
\i Output channel distortion correction
\i Determination of SEL phase offset
\i Waveform feedforward (triggered by an event receiver)
\i Piezoelectric actuator control
\i {\it in-situ} calibration of $8\pi/9$ notch filter
\i Input from the Beam-Based Feedback subsystem
\i Interlocks

These topics are all over the map in terms of available laboratory
code base and field experience, and associated technical risk.
Barring surprises, most of them should be solvable with a process of
integrating and testing existing code.  ``High Level Applications''
support is often a sizable fraction of that effort.

References:

Phase and Amplitude Stabilization for Superconducting Resonators,
Jean R. Delayen Ph.D. Thesis, 1978

%Digital Self-Excited Loop, John Musson {\it et al.}, 2009?
A Digital Self Excited Loop for Accelerating Cavity Field Control,
T. Allison {\it et al.}, Proceedings of PAC07, Albuquerque, New Mexico, USA
%http://accelconf.web.cern.ch/AccelConf/p07/PAPERS/WEPMS060.PDF

DSP Tutorial III, Avoiding Resource Overutilization in FPGAs,
L. Doolittle, LLRF13, Lake Tahoe, October 2013

Digital Low-Level RF Control Using Non-IQ Sampling,
Lawrence Doolittle {\it et al.}, LINAC 2006
\bye
